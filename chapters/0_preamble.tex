%---------------------------------------------------------------------------
% Table of contents

 \setcounter{tocdepth}{2}
 \tableofcontents
 \cleardoublepage

%---------------------------------------------------------------------------
% List of Figures

 % \addcontentsline{toc}{chapter}{List of Figures}
 % \listoffigures
 % \clearpage

%---------------------------------------------------------------------------
% List of Tables

 % \addcontentsline{toc}{chapter}{List of Algorithms}
 % \listofalgorithms
 % \clearpage

%---------------------------------------------------------------------------
% Abstract

\chapter*{Abstract}
 \addcontentsline{toc}{chapter}{Abstract}
This thesis focused on autonomous quadrotor landing on a moving platform.\\
The aerial robot employs a forward-looking camera to perform state estimation, and a down looking-camera
to detect and observe the landing platform carried by a mobile robot moving independently inside an arena. Measurements from the downlooking camera are combined with a proper dynamic model in order to estimate position and velocity of the moving platform.\\
The overall goal is to design a complete framework to perform the entire task: area exploration to look for the base; finding and approaching the platform, while keeping it within the camera's field of view; finally landing on it, minimizing the error in position and velocity.\\
The frameworks consists of several modules that perform different functions and collaborate together to complete the mission. All the computation run onboard, and so the quadrotor can perform this task fully autonomously.\\
The system is validated in the real world experiments: the vehicle successfully landed on the moving platform during outdoor flight tests.
 \cleardoublepage

%---------------------------------------------------------------------------
% Symbols

\chapter*{Nomenclature}\label{chap:symbole}
\addcontentsline{toc}{chapter}{Nomenclature}

\section*{Notation}
  \begin{tabbing}
    \hspace*{1.6cm}   \= \kill
    $\mathbf{J}$       \> Jacobian \\[0.5ex]
    $\boldsymbol{r}$  \> position of the frame $B$ with respect to frame $W$ \\[0.5ex]
    $\mathbf{T}_{WB}$  \> coordinate transformation from frame $B$ to frame $W$ \\[0.5ex]
    $\mathbf{R}_{WB}$  \> orientation of $B$ with respect to $W$ \\[0.5ex]
    %$_W\mathbf{t}_{WB}$\> translation of $B$ with respect to $W$, expressed in coordinate system $W$ \\[0.5ex]
    $\hat{\boldsymbol{w}_{WB}}$ \> skew symmetric matrix \\[0.5ex]
    $\boldsymbol{c}$  \> thrust vector with respect to frame $B$ \\[0.5ex]
    $\boldsymbol{g}$  \> gravity with respect to frame $W$ \\[0.5ex]
  \end{tabbing}
  
Scalars are written in lower case letters ($a$), vectors in lower case bold letters ($\mathbf{a}$) and matrices in upper case bold letters ($\mathbf{A}$).

\section*{Acronyms and Abbreviations}
  \begin{tabbing}
    \hspace*{1.6cm}  \= \kill
    RPG     \> Robotics and Perception Group \\[0.5ex]
    UAV     \> Unmanned Aerial Vehicle \\[0.5ex]
    UGV     \> Unmanned Ground Vehicle \\[0.5ex]
    MAV     \> Micro Aerial Vehicle \\[0.5ex]
    ROS     \> Robot Operating System \\[0.5ex]
    DoF     \> Degree of Freedom \\[0.5ex]
    IMU     \> Inertial Measurement Unit \\[0.5ex]
    EKF   \> Extended Kalman Filter \\[0.5ex]
    SVO   \> Semidirect Visual Odometry \\[0.5ex]
    MSF   \> Multi Sensor Fusion \\[0.5ex]
    MBZIRC    \> Mohamed Bin Zayed International Robotics Challenge \\[0.5ex]
  \end{tabbing}

\clearpage

%---------------------------------------------------------------------------
