\chapter{Landing State Machine}\label{chap:area_exploration}
This section describes the module that, based on the state estimation of the UAV and of the moving platform, decides in which state the framework is, and which is the desired state that the quadrotor must reach in order to complete the mission.\\
This module implements a state machine and the flow diagram can be seen in Fig.~\ref{fig:area_exploration_state_machine}.

\begin{figure}[!htbp]
    \centering
    \includegraphics[width=1\textwidth]{img/state_machine.pdf}
    \caption{Landing state machine flow chart.}
    \label{fig:area_exploration_state_machine}
\end{figure}

It consists of 5 main parts in each of which the MAV has to complete a particular task in order to proceed with the successive stage. The task is defined by a precise final state that the quad must reach, and this final condition is considered reached when the position of the quad is inside a sphere of radius $\rho_{reached}$ around the final position.\\

In the following we describe in detail all these stages and explain the computation that we perform in order to decide where the UAV must go and when a particular stage is considered concluded.


\section{Takeoff and searching for the base}
In this stage the quadrotor starts from a position near to the ground and has to explore a given area from high altitude until the platform is found.\\

At the beginning, the quad hovers close to the ground, then a vertical takeoff is performed until it reaches a given altitude $h$. This vertical maneuver is performed anytime the pipeline fails and we have to restart the state machine.\\
Given the area that must be explored to find the target (in our case it will be the arena in which the platform can move, see Fig.~\ref{fig:arenachallenge}), we calculate a list of way-points the UAV must reach in order to span the whole area.\\
As the quadrotor moves, the camera collects information from a large section of the area and the searching task of the target can be performed faster.\\
There are many ways to sample the way-points to explore the area, in our case we try to maximize the probability to find the moving platform so we are moving from one side to the other, along the main axis of the infinity-shape path.\\

As soon as the moving platform is found, the state machine proceeds with the next stage.

\section{Following the base}
In this stage the quadrotor has to follow the moving platform until we identify the right moment it has to start the landing maneuvers.\\

The MAV moves at high altitude, reaching the desire points given by the previous stage.  As soon as an estimation of the target state is available, the quad begins to follow the platform and performs the following computations in order to complete the task of this stage.

\subsection{Understand type of movement}
From the challenge description in Sec.~\ref{fig:arenachallenge}, we know that the car moves in a shape composed by straight lines and circumference sectors. We need to understand, at a given time, in which part the platform is: this information is important in order to calculate properly where the platform will be in $t$ seconds. Also, we want to perform the landing maneuver, when the platform is moving on a straight line.\\

To understand the trajectory of the moving base, we collect all the estimated positions of the base coming from the previous module, described in Chap.~\ref{chap:base_tracking}, and we perform a linear regression on the last $n$ estimations. The platform moves in a straight line if the linear regression is a good approximation of the data trends, otherwise it driving on a curve.\\

We have a series of $n$ points, each of these is consider as a pair of coordinates $(x_i,y_i)$, and we are searching for the best-fit line that can describe the data as a linear function: 
\begin{align}
y = mx + q
\end{align}
In our case, there are no real dependent and independent variables so we perform the following analysis considering before the coordinates $y_i$ as dependent, then solve the dual problem with $x_i$, and finally peaking the fit with smaller approximation error. \\
We want find the best parameters $m$ and $q$, and to do so we need to have some measures of quality to optimize. Unless all our $n$ points are already in a perfect line there will be an error between the value predicted by the line, and the observed dependent variable:
\begin{align}
e_i = y_i - (mx_i + q)
\label{theor_resid}
\end{align}

These differences are called residuals and what we want is to find the model that minimizes: 
\begin{align}
(m^*,q^*) = argmin \sum_{i=1}^{n}{e_i^2}
\end{align}
The model we find is the Least Squares Fit of the data. We define also the cumulative residual as: 
\begin{align}
e_{tot} = \sqrt{\sum_{i=1}^{n}{e_i^2}}
\end{align}

The parameters $m$ and $q$ of the model are found where $e_{tot}^2$ is minimized. To do so, the following first order condition must hold:
\begin{subequations}
\begin{align}
\frac{e_{tot}^2}{\partial m} = 0,\\
\frac{e_{tot}^2}{\partial q} = 0.
\label{eq:mandq1}
\end{align}
\end{subequations}

It is easy to demonstrate that the solution of Eq.~\eqref{eq:mandq1} is:
\begin{subequations}
\begin{align}
m^* &= \ddfrac{n\sum_{i=1}^{n}{x_iy_i} - \sum_{i=1}^{n}{x_i}\sum_{i=1}^{n}{y_i}}{n\sum_{i=1}^{n}{x_i^2} -( \sum_{i=1}^{n}{x_i})^2}, \\[10pt]
q^* &= \ddfrac{ \sum_{i=1}^{n}{y_i}}{n} - m\ddfrac{ \sum_{i=1}^{n}{x_i}}{n}.
\label{eq:mandq}
\end{align}
\end{subequations}

The platform moves on a straight line if the cumulative residual $e_{tot}$ is below a threshold $th_{line}$, while if the error is above $th_{curve}$ the base is traveling the circumference. If $th_{line} \leq e_{tot} \leq th_{curve}$ then the type of movement cannot be determinate, and we assume that the platform keeps moving with the same style found before.\\

To have a good interpretation of the data it is important to decide the three parameters $n$, $t_{line}$, $t_{curve}$ correctly:
\begin{itemize}
\item The first parameter $n$ is the number of samples to consider when we perform the linear regression. We chose it in order to consider poses that are along a curve with length:
\begin{align}
l_{curve} = \frac{\rho_8\pi}{4}. \label{eq:lengthcurve} 
\end{align}
We know the forward constant velocity of the car $v_{tan}$, so we can calculate the time it takes the platform to perform this curve:
\begin{align}
t_{curve} = \frac{l_{curve}}{v_{tan}}. \label{eq:timecurve} 
\end{align}

When we receive a pose at time $t_i$ we store it and we perform the linear regression with all the data stored in the interval $[t_i-t_{curve},t_i]$.

\item The threshold parameters are calculated considering that each measure is corrupted by an additive Gaussian noise with 0 mean and $\sigma_e^2$ variance: 
\begin{align}
\tilde{y_i} = \mathcal{N}(y_i,\sigma_e^2) .
\end{align}
When we perform the linear regression on the measured data, the average residual square is 
\begin{align}
\begin{split}
<\tilde{e_i}^2> &= <(\tilde{y_i} - (mx_i + q))^2>  \\[5pt]
&= <\tilde{y_i}^2 - 2\tilde{y_i}(mx_i + q) + (mx_i + q)^2>  \\[5pt]
&= <\tilde{y_i}^2> - 2<\tilde{y_i}>(mx_i + q) + (mx_i + q)^2 \\[5pt]
&=  \sigma_e^2 + y_i^2  - 2y_i(mx_i + q) + (mx_i + q)^2  \\[5pt]
&=  \sigma_e^2 + e_i^2 
\end{split}
\end{align}
Now we have to distinguish if the total approximation error is related to a straight line or a curve section:
\begin{itemize}
\item when we perform the linear regression on linear data the theoretical data are distributed as:
\begin{align}
\begin{cases}
x_i = x_i \\[5pt]
y_i = ax_i + b
\end{cases},
\end{align} 
so the theoretical residual, calculated using Eq.~\eqref{theor_resid}, is: 
\begin{align}
e_i = ax_i + b - (mx_i + q).
\end{align}
When we try to calculate the model parameter $m^*$ and $q^*$ with Eq.~\eqref{eq:mandq}  the result leads to:
 \begin{align}
\begin{cases}
m^* &= a  \\[5pt]
q^* &= b
\end{cases}.
\label{eq:mandqcirum}
\end{align}
So, obviously, the theoretical residual squares is:
\begin{align}
e_i^2  = 0,
\end{align} 
and the average residual squares on the measured data is:
\begin{align}
<\tilde{e_i}^2>  = \sigma_e^2.
\end{align} 
The parameter $th_{line}$ is then:
\begin{align}
th_{line} = \sqrt{\sum_{i=1}^{n}{\tilde{e_i}^2}} = \sqrt{\sum_{i=1}^{n}{\sigma_e^2}} = \sigma_e\sqrt{n}.
\end{align}
\item When we perform the linear regression on data along a circumference arch with radius $\rho$ and angles $\theta_i \in [\theta_1,\theta_2]$, the theoretical data are distributed as:
\begin{align}
\begin{cases}
x_i = \rho\cos{\theta_i}\\[5pt]
y_i = \rho\sin{\theta_i},
\end{cases}
\end{align} 
so the theoretical residual, calculated using Eq.~\eqref{theor_resid}, is:
\begin{align}
e_i = \rho\sin{\theta_i} - (m\rho\cos{\theta_i} + q)
\end{align}
To find $m^*$ and $q^*$ we use Eq.~\eqref{eq:mandq}, but we want a general approximation of these values. To do so, we have  to consider all the sums in the equations as integrals, using the relation:
\begin{align}
\lim_{n \to \infty} { \frac{b-a}{n} \sum_{i=0}^{n}{f(x_i)}} &=   \int_a^b{f(x )\mathrm  {d}x} \\[10pt]
 \sum_{i=0}^{n}{f(x_i)} &\simeq \frac{n}{b-a} \int_a^b{f(x )\mathrm  {d}x}
\label{eq:integralsandsums}
\end{align} 

So now if we calculate this approximation for our values we have:
\begin{align}
\begin{split}
\sum_{i=1}^{n}{x_iy_i} &= \sum_{i=1}^{n}{\rho^2\cos{\theta_i}\sin{\theta_i}}   \\
& \simeq  \frac{n}{\theta_2 - \theta_1}  \rho^2 \int_{\theta_1}^{\theta_2}{\cos{x}\sin{x} \mathrm  {d}x} \\
&=  \frac{n}{\theta_2 - \theta_1}  \frac{ \rho^2}{2} \Big[-\cos^2{x} \Big]_{\theta_2}^{\theta_2}
\end{split}
\label{eq:mandqintegralsoncircum1}
\end{align}
\begin{align}
\begin{split}
 \sum_{i=1}^{n}{x_i} &= \sum_{i=1}^{n}{\rho\cos{\theta_i}} \\
&\simeq  \frac{n}{\theta_2 - \theta_1}  \rho \int_{\theta_1}^{\theta_2}{\cos{x}\mathrm  {d}x} \\
&= \frac{n}{\theta_2 - \theta_1}  \rho \Big[\sin{x} \Big]_{\theta_2}^{\theta_2}
\end{split}
\label{eq:mandqintegralsoncircum2}
\end{align}
\begin{align}
\begin{split}
 \sum_{i=1}^{n}{y_i} &= \sum_{i=1}^{n}{\rho\sin{\theta_i}} \\
& \simeq  \frac{n}{\theta_2 - \theta_1}  \rho \int_{\theta_1}^{\theta_2}{\sin{x}\mathrm  {d}x} \\
&= \frac{n}{\theta_2 - \theta_1}  \rho \Big[-\cos{x} \Big]_{\theta_2}^{\theta_2}
\end{split}
\label{eq:mandqintegralsoncircum3}
\end{align}
\begin{align}
\begin{split}
 \sum_{i=1}^{n}{x_i^2} &= \sum_{i=1}^{n}{\rho^2\cos^2{\theta_i}} \\
& \simeq  \frac{n}{\theta_2 - \theta_1}  \rho^2 \int_{\theta_1}^{\theta_2}{\cos^2{x}\mathrm  {d}x} \\
&=  \frac{n}{\theta_2 - \theta_1}  \frac{ \rho^2}{2} \Big[ x+\cos{x} \sin{x}\Big]_{\theta_2}^{\theta_2}
\label{eq:mandqintegralsoncircum4}
\end{split}
\end{align}

In our case we consider pieces of curve with length $l_{curve}$ defined in Eq.~\ref{eq:lengthcurve}, that correspond to a circumference arch with:
\begin{align}
\rho = \rho_8  \ \ \ \ \ \ \ \ \ \ \ \ \ 
\theta_i \in \Big[0,\frac{\pi}{4}\Big]
\label{eq:valuessircum}
\end{align}

We can now calculate the approximate values of $m*$ and $q*$ using Eq.~\eqref{eq:mandq} and Eqs.~\eqref{eq:mandqintegralsoncircum1} - \eqref{eq:valuessircum}:
\begin{subequations}
\begin{align}
\begin{split}
m^* &= \ddfrac{n\rho_8^2\frac{n}{\pi} - \rho_8\frac{n2\sqrt{2}}{\pi} \rho_8\frac{n2(2-\sqrt{2})}{\pi} }{n\frac{n\rho_8^2(2+\pi)}{2\pi} -(\rho_8\frac{n2\sqrt{2}}{\pi})^2}\\
 &= \ddfrac{2\pi - 16\sqrt{2} + 16}{\pi^2 + 2\pi - 16}
 \end{split}
 \end{align}
 \begin{align}
 \begin{split}
q^* &= \ddfrac{ \rho_8\ddfrac{n2(2-\sqrt{2})}{\pi} }{n} - m\ddfrac{ \rho_8\frac{n2\sqrt{2}}{\pi}}{n}\\
 &= \rho_8 \ddfrac{4-2\sqrt{2}(m+1)}{\pi} = \rho_8\bar{q} 
  \end{split}
\end{align}
\label{eq:mandqcirum}
\end{subequations}
Now we have to calculate the theoretical residual square $e_i^2$, but in this case we can compute the algebraic average residual square $<e_i^2>$, using again the approximations \eqref{eq:mandqintegralsoncircum1} - \eqref{eq:mandqintegralsoncircum4}, the values calculate in \eqref{eq:mandqcirum}, and averaging over the $n$ samples we consider:
\begin{align}
\begin{split}
<e_i^2> &=  \frac{1}{n}\sum_{i = 1}^{n}{\Big(\rho_8\sin{\theta_i} - (m\rho_8\cos{\theta_i} + q)\Big)^2}\\
%&= \frac{1}{n}\sum_{i = 1}^{n}{\rho_8^2\sin^2{\theta_i} - 2\rho_8\sin{\theta_i}(m\rho_8\cos{\theta_i} + q) + (m\rho_8\cos{\theta_i} + q)^2}\\
& =  \frac{1}{n}\sum_{i = 1}^{n} \rho_8^2 \xi = \rho_8^2 \xi 
\end{split}
\end{align}
%In particular, with $\theta$ having values like in \eqref{eq:valuessircum} we lead with:
%\begin{align}
%\xi &= \frac{\pi - 2 + m^2(\pi+2) + 2\pi\bar{q}^2 - 4m -8\bar{q}(2-\sqrt{2}+ m\sqrt{2})}{2\pi}
%\end{align}
Finally we calculate 
\begin{align} 
<\tilde{e_i}^2>  = <e_i^2> + \sigma_e^2 = \rho^2 \xi  + \sigma_e^2.
\end{align} 
The parameter $th_{curve}$ is then:
\begin{align}
th_{curve} = \sqrt{\sum_{i=1}^{n}{\tilde{e_i}^2}} = \sqrt{\sum_{i=1}^{n}{\rho^2 \xi  + \sigma_e^2}} = \sqrt{n}(\sigma_e + \rho\sqrt{\xi}).
\end{align}
\end{itemize}
\end{itemize}

Figure \ref{fig:error_regression} shows the typical evolution of the total residual during this first stage: the different phases of linear and circular movement can be detect in the graph. Furthermore, the points of regime change can be seen both in Fig.~\ref{fig:error_regression}  and in Fig.~\ref{fig:error_regression_map} in which also all the estimated positions of the base are plotted.

\begin{figure}[!htbp]
    \centering
    \includegraphics[width=0.9\textwidth]{img/following_platform_for_long_position_base_error.pdf}
    \caption{Evolution of the total residual during this first phase (in blue). The vertical lines are the real moments in which the car changes movement types: green a linear phase starts, red a circular phase begins. The horizontal lines are the thresholds for the detection of the two different phases. The crosses are the moments in which the algorithm understands the change.}
    \label{fig:error_regression}
\end{figure}
\begin{figure}[!htbp]
    \centering
    \includegraphics[width=0.8\textwidth]{img/following_platform_for_long_map_simple.pdf}
    \caption{Map of the estimated positions of the platform in blue. The crosses are the moments in which the algorithm understands the change. Red crosses from line to curve. Green crosses from curve to line.}
    \label{fig:error_regression_map}
\end{figure}

We can see from these graphs that with the proposed method is possible to distinguish clearly the period of time in which the platform is traveling along a line or along a curve, and it is also finds the switching points with good accuracy.\\ 
The major drawback of this method is that it is necessary an amount of time equal to $t_{curve}$ to understand that the platform switched motion regime. As a matter of fact, to detect a straight line movement we need that all the positions, considered in the linear regression, lay on the line.  \\
If  $th_{line}$ and  $th_{curve}$ are too close, it is always possible to consider a curve with longer length $l_{curve}$: this will increase the latter threshold with respect to the former, but will also increase the time $t_{curve}$ to understand the type of movement.

\subsection{Calculate future positions of the moving platform}
If the platform regime of movement at a specific time is known, we can estimate where it will be after $t_s$ seconds and proceed with the following stages when it starts a straight portion of the trajectory.\\
Thanks to the algorithm described before, we can estimate that if at time $t_0$ the car is at position $(x_0,y_0)$ with a direction angle of $\theta_0$ and forward velocity of $v_{tan}$, at time $t_1 = t_0 + t_s$ the car will be at position $(x_1,y_1)$ with an angle  $\theta_1$, and it has traveled  $v_{tan}t_s$.
\begin{itemize}
\item When no regime is found (at the begging) or when a line movement is detect, the predicted state is:
\begin{align}
\begin{cases}
x_1 &= x_0 + v_{tan}t_s\cos{\theta_0}\\[5pt]
y_1 &= y_0 + v_{tan}t_s\sin{\theta_0}\\[5pt]
\theta_1 &= \theta_0
\end{cases}.
\label{eq:line_future_pose}
\end{align}
\item When a movement on the circumference is detected, we have to perform some calculations in order to find the final state of the platform.\\ 
First, we use the relation:
\begin{align}
l_{curve} = v_{tan}t_s = \rho_8|\beta_s|,
\end{align}
where $\beta_s$ is the angle spanned by the platform in $t_s$ seconds:
\begin{align}
|\beta_{s}| &= \frac{v_{tan}t_s}{\rho_8}.
 \label{eq:anglespanned}
\end{align}
The final angle will be:
 \begin{align}
 \theta_1 = \theta_0 + \beta_s.
 \label{eq:anglefinal}
 \end{align}

Referring to Fig.~\ref{fig:chord} we can calculate that the segment connecting $(x_0,y_0)$ and $(x_1,y_1)$ has:
\begin{itemize}
\item direction $\theta_{chord}$ found as bisection between $\theta_0$ and $\theta_1$:
 \begin{align}
 \theta_{chord} = \theta_0 + \frac{\theta_0 + \theta_1}{2} ;
 \label{eq:anglechord}
 \end{align}
\item length $l_{chord}$, found with the chord theorem:
\begin{align}
 l_{chord} = 2\rho_8\sin{\frac{|\beta_s|}{2}}.
  \label{eq:lengthchord}
 \end{align}
\end{itemize}
\begin{figure}[!htbp]
    \centering
    \includegraphics[width=0.5\textwidth]{img/chord.pdf}
    \caption{In red, the position of initial state at time $t_0$. In green, the final estimate state at time $t_1$. In yellow, the chord between the two states with length $l_{chord}$ and orientation $\theta_{chord}$.}
    \label{fig:chord}
\end{figure}

Now we have all the elements to calculate the final point $(x_1,y_1)$, but in order to properly find it we have to resolve another last problem. Both $\beta_{s}$ and  $-\beta_{s}$ span a curve of length $v_{tan}t_s$, and due to the symmetry of our trajectory is impossible to know beforehand which angle is the right one. \\
What we can do is calculate both the two possible final states using Eqs.~\eqref{eq:anglespanned}, \eqref{eq:anglechord} and \eqref{eq:lengthchord}:
\begin{align}
\begin{cases}
x_1^a &= x_0 + l_{chord}\cos{\Big(\theta_0 + \frac{|\beta_s|}{2}\Big) }\\[5pt]
y_1^a &= y_0 + l_{chord}\sin{\Big(\theta_0 + \frac{|\beta_s|}{2} \Big)}\\[5pt]
\theta_1^a &=  \theta_0 + |\beta_s|
\end{cases},
\end{align}
\begin{align}
\begin{cases}
x_1^b &= x_0 + l_{chord}\cos{\Big(\theta_0 - \frac{|\beta_s|}{2}\Big) }\\[5pt]
y_1^b &= y_0 + l_{chord}\sin{\Big(\theta_0 - \frac{|\beta_s|}{2}\Big) }\\[5pt]
\theta_1^b &=  \theta_0 - |\beta_s|
\end{cases}.
\end{align}

In order to understand which one is the correct state, we can calculate the distance between the two possible final points and a point of the trajectory estimated at time $ t_{-\alpha} < t_0$. The state with smaller distance will be the right final state, because the wrong one leads to a position further away.\\

Figure \ref{fig:sequence_find_next_position_circumference} summarizes the passages we perform to find the right final state explained above.

\begin{figure}[!htbp]
  \centering
   \begin{subfigure}[b]{0.45\textwidth}
        \includegraphics[width=\textwidth]{img/circular_movment1.pdf}
        \caption{Red cross with arrow: state a $t_0$. Red arrow: current velocity vector. Red cross: previous position a $t_{-\alpha}$.}
        \label{fig:one}
   \end{subfigure}\hfill
   \begin{subfigure}[b]{0.45\textwidth}
        \includegraphics[width=\textwidth]{img/circular_movment2.pdf}
        \caption{Black line: direction of the velocity vector, with angle $\theta_0$. We have symmetry with respect to this axes.}
        \label{fig:two}
   \end{subfigure}
   
   \begin{subfigure}[b]{0.45\textwidth}
        \includegraphics[width=\textwidth]{img/circular_movment3.pdf}
        \caption{Blue lines: real and symmetric path. We do not know which of the two trajectories is correct.}
        \label{fig:three}
   \end{subfigure}\hfill
    \begin{subfigure}[b]{0.45\textwidth}
        \includegraphics[width=\textwidth]{img/circular_movment4.pdf}
        \caption{Yellow arrows: estimated future angles $\theta_1^a$ and $\theta_1^b$ that the base will assume at time $t_1$.}
        \label{fig:four}
   \end{subfigure}
   
    \begin{subfigure}[b]{0.45\textwidth}
        \includegraphics[width=\textwidth]{img/circular_movment5.pdf}
        \caption{Green arrows: segment from state at $t_0$ and possible states at $t_1$: with angles $\theta_{chord}^a$  and $\theta_{chord}^b$ and length $l_{chord}$ }
        \label{fig:five}
   \end{subfigure}\hfill
    \begin{subfigure}[b]{0.45\textwidth}
        \includegraphics[width=\textwidth]{img/circular_movment6.pdf}
        \caption{Green crosses: future possible states of the platform at positions $(x_1^a,y_1^a)$ and $(x_1^b,y_1^b)$ .}
        \label{fig:six}
   \end{subfigure}
   
    \begin{subfigure}[b]{0.45\textwidth}
        \includegraphics[width=\textwidth]{img/circular_movment7.pdf}
        \caption{Dark grey lines: distances from previous position at $t_{-\alpha}$ and possible future positions at $t_1$. }
        \label{fig:seven}
   \end{subfigure}\hfill
    \begin{subfigure}[b]{0.45\textwidth}
        \includegraphics[width=\textwidth]{img/circular_movment8.pdf}
        \caption{Green cross with arrow: future state selected taking the position with minimum distance.}
        \label{fig:eight}
   \end{subfigure}
  \caption{The sequence of passages computed in order to select the future position when the platform is moving on the circumference.}
  \label{fig:sequence_find_next_position_circumference}
\end{figure} 
\end{itemize}

At this point we can use the predicted position of the platform to control the quadrotor following the base.\\

Figure \ref{fig:map_waypoints} shows the points in which the algorithm calculates where the quadrotor should go in order to follow the moving car. It is noticeable the subdivision of point calculates with the linear model (green stars) and with the circular one (red stars).
% and also that the algorithm needs some time, $t_{curve}$, to understand the switch of movement regime and this can be seen in the delay in changing between calculation models.

\begin{figure}[!htbp]
    \centering
    \includegraphics[width=0.8\textwidth]{img/following_platform_long_map_waypoints.pdf}
    \caption{Map of the estimated positions of the platform in blue.  Stars positions in which the quadrotor should go to following the base until a proper moment to proceed with the follow stage is detected. The green points are calculated with the linear model, while the red ones with the circular one. }
    \label{fig:map_waypoints}
\end{figure}

\subsection{Select moment to land}
When the vehicle is following the platform, we have to detect the right moment to proceed with the other stages.\\
The right moment to start the landing maneuver is at the start of a line segment:
\begin{itemize}
\item if we detect the base and we understand that it is moving in the circumference (where we cannot land), we have to follow the base and wait until we detect a change in the regime from curve to line. At this point we can proceed with the following stages.\\
Proceed with the landing when we detect a passage between circumference and line movement can be risky if the platform is moving fast: as a matter of facts we know that the length of the straight line is $2\rho_8$ but we understand that the platform is moving straight after $l_{curve} = \frac{pi}{4}$ after it finished the curve, so the quad has just 
\begin{align}
t_{landing} = \frac{\rho_8(2-\frac{pi}{4})}{v_{tan}}
\end{align}
seconds to perform the land with the platform moving in line. If the velocity of the platform is too high this time may not be sufficient.
\item If we detect the base for the first time and we understand that it is moving in a straight line it is very risky to perform the landing maneuver. As a matter of fact, we do not know when the platform started the line segment and in how many seconds it will begin the curve section. The platform can be almost at the end of line and it is about to start the circumference, in this case we do not have time to perform the entire landing maneuver.\\
What we do is following the car and waiting when it changes movement regime, from line to curve. At this point we can calculate where the next change point, from curve to line, will be and start the landing at that point.\\
In this case we will have more time to complete the landing, because the quadrotor starts the maneuver at the begin of the line, and not after $l_{curve}$ as in the previous case. Also it can proceed with the first stage of the the landing even a little before the start of the segment, so:
\begin{align}
t_{landing} \geq \frac{2\rho_8}{v_{tan}}
\end{align}.

In order to calculate where the future changing point will be, we must perform some computations:
\begin{itemize}
\item we know the orientation $\theta_{line}$ of the straight line just finished: the platform just changed from line regime to curve and we have saved the inclination $m$ of the best linear approximation found while it was moving in line;
\item given the point of change between line and curve, the future point will be in the circumference after an angle of $\big | \frac{3\pi}{2} \big |$;
\item from Eqs.~\eqref{eq:anglechord} and \eqref{eq:lengthchord} we know that the segment connecting the change point and the future intersection point has length $\sqrt{2}\rho_8$ and angle $\theta_{line} \pm \frac{3\pi}{4} $;
\item we can apply the same method described before to find the two possible intersection points:
\begin{align}
\begin{cases}
x_{intersection}^a &= x_{changing} + \sqrt{2}\rho_8\cos{\Big(\theta_{line} + \frac{3\pi}{4}\Big) }\\[5pt]
y_{intersection}^a &= y_{changing} + \sqrt{2}\rho_8\sin{\Big(\theta_{line} + \frac{3\pi}{4}\Big) }\\[5pt]
\theta_{intersection}^a &=  \theta_{line} + \frac{3\pi}{2}
\end{cases},
\end{align}
\begin{align}
\begin{cases}
x_{intersection}^b &= x_{changing} + \sqrt{2}\rho_8\cos{\Big(\theta_{line} - \frac{3\pi}{4}\Big) }\\[5pt]
y_{intersection}^b &= y_{changing} + \sqrt{2}\rho_8\sin{\Big(\theta_{line} - \frac{3\pi}{4}\Big) }\\[5pt]
\theta_{intersection}^b &=  \theta_{line} - \frac{3\pi}{2}
\end{cases},
\end{align}
and select the right one with minimum distance with the current estimate position of the platform.
\end{itemize}


Figure~\ref{fig:sequence_find_next_intersection} summarizes all the passages we perform to find the right intersection point just explained.

\begin{figure}[!htbp]
  \centering
   \begin{subfigure}[b]{0.45\textwidth}
        \includegraphics[width=\textwidth]{img/intersection_1.pdf}
        \caption{Red cross with arrow: state a $t_0$. Red arrow: current velocity vector. Red cross: changing point from line to curve.}
        \label{fig:one}
   \end{subfigure}\hfill
   \begin{subfigure}[b]{0.45\textwidth}
        \includegraphics[width=\textwidth]{img/intersection_2.pdf}
        \caption{Black line: direction of the line sector just finished. The direction is taken as the slope of the best linear fit found in the previous regime.}
        \label{fig:two}
   \end{subfigure}
   
   \begin{subfigure}[b]{0.45\textwidth}
        \includegraphics[width=\textwidth]{img/intersection_3.pdf}
        \caption{Blue lines: real and symmetric path. We do not know which of the two trajectories is correct. Yellow crosses: in both the path we can calculate the future intersection point.}
        \label{fig:three}
   \end{subfigure}\hfill
    \begin{subfigure}[b]{0.45\textwidth}
        \includegraphics[width=\textwidth]{img/intersection_4.pdf}
        \caption{Dark grey lines: distances from current position and the two possible future intersections. Both are eligible becouse of the symmetry of the trajectory.}
        \label{fig:four}
   \end{subfigure}
   
    \begin{subfigure}[b]{0.45\textwidth}
        \includegraphics[width=\textwidth]{img/intersection_5.pdf}
        \caption{Yellow cross with arrow: future intersection point selected taking the position with minimum distance from the current state. }
        \label{fig:five}
   \end{subfigure}
  \caption{The sequence of passages computed in order to select the future intersection point where the platform will start the movement in line.}
  \label{fig:sequence_find_next_intersection}
\end{figure} 
\end{itemize}

At this point the quad is following the moving platform, but as soon as the base is close to the future changing point, the quad can proceed with the next stage.\\

Figure \ref{fig:map_intersections} shows where the algorithm calculates the future changing points (yellow crosses), based on the current one (red crosses). When the platform is in a neighborhood of these points is the right moment for the UAV to approach the base.

\begin{figure}[!htbp]
    \centering
    \includegraphics[width=0.8\textwidth]{img/following_platform_normal_map_intersection.pdf}
    \caption{Estimated positions of the platform in blue.  Yellow crosses are the position where the quadrotor should go in order to intersect the platform when is about to start a line phase.}
    \label{fig:map_intersections}
\end{figure}

\section{Approaching the base}
In this stage the quadrotor has to decrease its altitude keeping the platform in the field of view until a better state estimation (from the low altitude EKF) of the platform itself is available.\\

The quadrotor is following the base at high altitude. As soon as the platform starts a straight line sector, the UAV has to approach it reducing its altitude and keeping the target in the field of view of the camera.\\
In this phase, the desired final $x,y$ coordinates of the quadrotor are calculated with the same equations of the previous stage when a line movement of the platform is detected Eq.~\eqref{eq:line_future_pose}. 
The main difference are the $z$ coordinate and the $x,y$ final velocities that the quad has to assume.\\
We set the final velocity identical to estimated velocity of the base:
\begin{align}
\begin{cases}
vx_1 &= v_{tan}\cos{\theta_0}\\[5pt]
vy_1 &= v_{tan}\sin{\theta_0}
\label{eq:finalstavelocity}
\end{cases}
\end{align}
While the final altitude is computed as:
\begin{align}
z_1 = \alpha z_0 + (1 - \alpha) z_{0,target},
\label{eq:finalz}
\end{align}
where $\alpha <  1$ is a parameter to be tuned in order to have a right balance between time duration of this stage and aggressiveness of the maneuver.\\
As a matter of fact, if the quad approaches too quickly the target, it is very easy to lose the platform from the field of view. of the camera: the closer we are to the base the less area we can cover with the camera, so the more precise we must be in order to still have tracking of the target.\\
For example, if we set the final state of the quad equal to the state estimation of the moving base we have from high altitude ($\alpha = 0$), we are directly performing the landing on the base. In this case the UAV approaches a final position that may not be the one in which the platform will be: because the high altitudes EKF does not give a really precise estimation of the platform the quadrotor reaches a wrong place and it can loose the platform. \\
It is necessary to have some intermediate stages in which the quad gets closer to the platform, so it can refine and correct the final target, but at the same time the area spanned by the camera is large enough to detect the base even if the quad is not in the right pose.\\
If we loose the platform, this stage fails, and we have to takeoff again until the platform is in the field of view again.\\
Figure  \ref{fig:approach_platform} summarizes a situation in which approaching the platform too aggressively can lead to the loss of tracking, while if we have intermediate steps we can recover from the estimation error we had.\\ 

With this approach the platform can stay in the field of view of the camera much easier:  the quad is going ahead of the base, with a direction that is the same of the target, and correcting the estimate position of the moving platform at each step. \\
As soon as we are close enough to have a more precise state estimation from the low altitude EKF, we proceed with the next stage.

\begin{figure}[!htbp]
 \centering
   \begin{subfigure}[b]{1.0\textwidth}
     \includegraphics[width=\textwidth]{img/approach_platform_lose.png}
        \caption{If we set $z_1$ too close to the platform we can loose the tracking because the position estimation of the moving base is not good enough from high altitude.}
        \label{fig:loose_platform}
   \end{subfigure}
 \end{figure}
\begin{figure}[!htbp] \ContinuedFloat  
   \begin{subfigure}[b]{1.0\textwidth}
     \includegraphics[width=\textwidth]{img/approach_platform.png}
        \caption{With intermediate steps, instead, it is more difficult to loose the platform and we can correct the state estimate.}
        \label{fig:not_loose_platform}
   \end{subfigure}
    \caption{A situation in which intermediate steps, while approaching he moving platform from high altitude, are necessary to not loose the platform.  }
    \label{fig:approach_platform}
\end{figure}
 
\section{Aligning with the base}
In this stage the quadrotor flies close to the platform and, before proceeding with the landing maneuver, it has to align its direction with the one of the base to be in the best position to proceed with the final step.\\

When a state estimate of the base from the low altitude EKF is available the quad can rely on this information to align its movement to the base. The estimate is good enough to predict the future states of the moving platform even if we loose tracking. As a matter of fact, in this stage we know that the platform is moving in a straight line, and we now its initial position, direction and velocity, so it is easy to predict where it will be in $t$ seconds.\\

Given the current state of the platform and of the UAV we can calculate the distance $d$ between the two. This segment can be covered by the quadrotor with an maximum relative speed given by:
\begin{align}
v_{max,rel} &= v_{max,quad} - v_{base},
\label{eq:relative_vel}
\end{align}
in a time:
\begin{align}
t &= \frac{d}{||v_{max,rel}||}.
\end{align}
We can see from Eq.~\eqref{eq:relative_vel} that if the maximum velocity of the quadrotor  is equal in magnitude to the velocity of the platform, the time $t$ to reach the platform is infinite. Intuitively speaking, this means that the quadrotor has to be able to reach higher linear velocities than those of the moving platform.\\

In this period of time we know that the platform is moving to a different place, and we have to predict where it will be in $t$ seconds.\\
To predict the future position we use the same model used in the EKF Eq.\eqref{eq:equation_nonholonomic_discrete}. With this model, we do not just predict where the platform will be after $t$ seconds, but we estimate its state for discrete moments in a window of time around $t$: $[t-t_1,t+t_2]$. This way we have a set of possible future candidate positions to execute the landing maneuver.\\
It is also possible to bring the quad ahead of the platform simply requiring to reach one of these candidate states in a shorter time.\\

This window of future state estimates is used by the trajectory generator module to calculate different possible alternatives to reach the platform.\\
In this stage, the final target of the quadrotor will be one of these candidate state with $x,y$ position and velocity equal to the one of the platform, and $z$ position set to a proper height $h_{align}$.\\

\begin{figure}[!htbp]
    \centering
    \includegraphics[width=0.8\textwidth]{img/prediction_platform.pdf}
    \caption{The scheme synthesizes the algorithm performed in this stage to find the possible final states the quadrotor should reach in order to intercept the base. }
    \label{fig:align_platform}
\end{figure}

As soon as the quadrotor reaches the final target state, in which it is aligned with the moving platform we proceed with the final stage.

\section{Landing on the base}
In this final stage the quadrotor is very close to the platform and is moving in the same direction. It has to decrease its altitude until it touches the platform and then switch the motors off in order to conclude the whole task.\\

The way to calculate the final target is the same used in the previous stage of the state machine: we predict possibles intersection points that can be reached by the quadrotor, and the final states of the UAV will be one of these predictions.\\
It is possible also to add a velocity in the $z$ direction in order to have a slightly more aggressive vertical landing.\\
Furthermore, because the visual odometry can fail when we are really close to the platform, we introduce the possibility of a blind landing, in which we are not using the state estimation of the quadrotor to control the UAV, but we control the thrust open loop. In this case, when we are at $h_{blind}$ over the platform, we start to apply a thrust $c_{quad}$ smaller than the gravity $g$, such that $||c_{quad}|| < g$ in order to decrease the altitude of the quadrotor until it touches the platform.\\
More specifically, in this phase, the quad is moving along the z axis with the following equation of motion:
\begin{align}
\begin{split}
z(t) = z_{quad,0} + v_{z,quad,0}t + \frac{(c_{quad,z} - g)t^2}{2}.
\label{eq:z_dynamics}
\end{split}
\end{align}
If the quad starts from $z_{quad,0} = h_{blind}$ after a time $t_{blind}$ we want $z(t_{blind}) = 0$. Furthermore, we consider the initial velocity $v_{z,quad,0}$ to be really little, so it can be neglected. We can then easily calculate the time $t_{blind}$:
\begin{align}
\begin{split}
t_{blind} = \sqrt{\frac{2h_{blind}}{g-c_{quad,z}}}.
\label{eq:z_dynamics}
\end{split}
\end{align}

We know that in this time $t_{blind}$ the platform will change its position. In particular it is moving on a straight line, so its coordinates will be :
\begin{align}
\begin{cases}
x_{base}(t_{blind}) &= x_{base}(0) + v_{tan}t_{blind}\cos{(\theta_{base})}\\[5pt]
y_{base}(t_{blind}) &= y_{base}(0) + v_{tan}t_{blind}\sin{(\theta_{base})}
\end{cases}.
\label{eq:future_pose_blind}
\end{align}

Therefore, we know that if we have selected the coordinates $\tilde{x},\tilde{y}$ as intersection points where we will start the blind landing over the platform, the quad must be at these coordinates $t_{blind}$ seconds before the platform in order to properly land over it. Notice that if $h_{blind} \rightarrow 0$ also $t_{blind} \rightarrow 0$ and so the quad arrives at the intersection point with the moving base.\\

\begin{figure}[!htbp]
    \centering
    \includegraphics[width=0.8\textwidth]{img/blind_landing.pdf}
    \caption{The scheme synthesizes the concept of the final blind landing. }
    \label{fig:align_platform}
\end{figure}

In order to detect when the quad hits the platform we check the measurements from the IMU. This provides different measurements, among which the value of the linear accelerations along the 3 axes. Using this data we can calculate the magnitude of the acceleration and we know that, when the UAV hits a surface, this quantity is showing a big pick, so with a simple threshold on the acceleration norm we can detect when the quadrotor has landed.\\
This solution does not take in account that data from the IMU are usually corrupted by noise (see Sec.~\ref{subsec:acceleration}) and so we could confuse a noisy measurement with a bump. \\
To make this detection more robust we can filter the data with a low pass filter:
\begin{align}
imu_{filt}(t_k) =  (1-e^{-\frac{t_k-t_{k-1}}{\tau_{imu}}})imu_{raw}(t_k) + e^{-\frac{t_k-t_{k-1}}{\tau_{imu}}} imu_{filt}(t_{k-1})
\label{eq:imu_filtered}
\end{align} 
The filter eliminates the noise but it slows down the response to changes, so the detection of the bump is done with some delay: the parameter $\tau_{imu}$ decides the cut frequency of the filter, so tuning this quantity can lead to filtered data with the right trade-off between smoothness and sensibility to changes.\\  
Using the absolute value of the acceleration does not take in account that the quadrotor could assume very high accelerations while is performing a normal flight, and so this acceleration could exceed the threshold and be detected as a landing on the platform.\\
A solution to have a robust and fast detector of the hit is to compare the raw data from the IMU with the filtered one: only the bump creates a big change that the filter cannot follow instantaneously, so the difference between the two version of the data will be very high only in this case.\\
Figure \ref{fig:imu_landing} shows the data used by the bump detector in order to find when the quad is touching the platform and switch off the motors.

\begin{figure}[!htbp]
    \centering
    \includegraphics[width=0.7\textwidth]{img/imu_landing.pdf}
    \caption{Data used by the bump detector: difference between the norm of the raw data from the IMU and its filtered version. The difference grows really fast only when the UAV hits the surface.}
    \label{fig:imu_landing}
\end{figure}
 
If something goes wrong and in this phase the quadrotor reaches a $z_{quad} < z_{base}$, then the landing failed and we have to takeoff again, resetting the state machine.
