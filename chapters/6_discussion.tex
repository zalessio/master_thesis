\chapter{Discussion}\label{chap:discussion}
This thesis proposes an algorithm to autonomous landing for a quadrotor on a moving platform using only onboard sensing and computing.
%The approach presented is combining the robustness of predicting and replanning methods with the possibility of onboard computation.\\
We demonstrated the functionality of this framework both in simulation and in real word experiments achieving good results in therms of robustness and precision.\\

All the computations are onboard, and this feature ensures the framework to be robust to delays and lost of connection with other equipment, very likely in a large environment as the one in MBZIRC.\\
Furthermore, the replanning, in an Model Predictive Control style, is very good to correct estimation errors, which can happen when flying at high speed and using only onboard sensors. On the other hand, when the initial condition of the quadrotor are far from the hovering state, the trajectory generator appears not really robust and, sometime, it fails to find a feasible path to complete the task.\\
Finally, the Extended Kalman Filters developed to estimate the state of the moving base, have very good results both at high and low altitudes. They provide a precise state estimate for a platform that is moving with any type of smooth trajectories.\\

Summing up, the developed framework to autonomous landing on a moving platform is robust, precise and very general: it can be applied to any kind of UAVs (equipped with a computational unit, an Inertial Measurement Unit and a camera), and can track any kind of moving vehicles (as long as they are provided with a suitable landing platform).

\section{Conclusion}\label{sec:conclusion}
In this thesis we presented a complete framework to allow a quadrotor to find, approach and land on a moving platform. We explained all the modules that make up the system, showing in detail the computation we perform in order to complete the assigned task. \\
A nonlinear estimator for the state of the platform was developed. It uses the information given by the non-holonomic dynamics model merged with the data provided by some measurement algorithms able to detect and track the moving target, based on images from a camera.\\
Furthermore, we developed a state machine that, at each moment, provides a task that must be completed by the quadrotor. This module bases its decisions on the history of tasks completed in the past, the current states of the quadrotor and of the moving platform.\\
Finally, we used a trajectory generator able to correct errors related to not precise final conditions or wrong trajectory tracking. At each control loop this module checks if a new trajectory is necessary and, if it is the case, provides the system with a new path to follow in order to complete the current task.\\
A  state estimation for the quadrotor, based on visual odometry, and a flight controller were integrate in the system to complete the framework.\\

Several experiments were carried in simulation and in the real world to demonstrate the functionality of our system. From these experiments we proved robustness of the framework up to a certain velocity of the quadrotor, after which, at higher speeds, the modules relative to the trajectory generation and to the state estimation of the quadrotor, might fail.\\
The flight tests showed satisfactory performance, but in general we can say that further work needs to be done in order to achieve results that can be used in the MBZIRC challenge.

\section{Future Work}\label{sec:future_work}
There are several improvements that can be done to this frameworks. On the one hand, we have some problems related to the quadrotor state estimation, that is not always robust, and with the trajectory generator algorithm that sometime fails to find a feasible trajectory. On the other hand, we have to change some aspects of the framework in order to reach the goals given by the MBZIRC challenge.\\
In the following we describe possible solutions to tackle these issues.

\subsection{SVO with multiple cameras}
Since the quadrotor is provided with more than one camera, we can use the images from all of them to estimate the pose of the quadrotor with SVO. This way we have a more robust and accurate state estimate. Furthermore, using more then one camera, we have low probability that  the visual odometry fails in the same time for all of them, so we can recover automatically when there are some problems with the images of one camera.

\subsection{State estimation with multiple sensors}
It is possible to fuse multiple sensors to have a more robust and precise state estimation. As a matter of fact, MSF can easily combine different sources of data and filtering them with IMU information.
The main two sensors we can add for this upgrade are:
\begin{itemize}
\item GPS: it gives a 3D absolute position with low accuracy, available in outside environment. It can be useful to have a state estimate even when the visual odometry fails: with this setup, the quadrotor can be able to reinitialize autonomously the visual odometry without human involvement. The uncertainty related to GPS measure is higher with respect to the one from SVO, but MSF takes in account this information and filters the data in the right way.
\item Teraranger \cite{teraranger}: it is a distance sensor which can operate both in indoor and outdoor environment. It is very light and can be really useful to have an estimation of the height of the quadrotor. It is in fact well known, GPS systems have much more error in the $z$ component, so the data from this sensor can correct the wrong estimations from the sensors. 
\end{itemize}

\subsection{Control}
As described before, the trajectory generator has some issues that must be solved  (See Sec. \ref{sec:trajectory_problem}). The approach to solve these problems can be:
\begin{itemize}
\item make the flight controller more sensitive: right now the replanning does not work because the first desired state of the trajectory is too close to the current state to generate a correct control action. Making the controller more response at little variation can solve this problem. A method to increase the sensitivity is tuning the controller gains, but this can lead to a unstable behavior, so further studies must be done.
\item Change both the trajectory and the contorller: implementing a new controller, like a LQR controller \cite{wiki_lqr}, that takes into account both the dynamics of the quadrotor and the platform, and directly calculate the control actions necessary to arrive in a certain final state. In this case the state machine should not predict in advance where the platform will be in $T$ seconds, because the LQR controller is using the dynamics of the platform in the state equations, and it is already taking in account how the platform position will change in time. The main problem with this type of controller is tuning the weights in the cost function to have a nice and smooth flight. \\ 
Furthermore we can implement a continuous replanning of the control actions of the LQR, leading with an MPC framework. This solution, as described in the introduction of the thesis, is really computationally expensive and before implementing it, we must understand if it can run onboard on our quadrotor.
\end{itemize} 

\subsection{Smart way-points sampling during the exploration}
Right now, during the exploration stage of the state machine, the quadrotor is following a precomputed set of way-points chosen such that the whole arena is spanned.\\
A future possible upgrade is to implement a particle filter \cite{particlefilter} that recursively estimate the positions in which is more probable to find the moving platform, and set one of these poses as target that the quadrotror must reach during the exploration stage. 
This set of possible locations for the moving platform is updated while the quadrotor is flying, such that the time to find the moving platform is minimized.\\

\subsection{Cross detector}
In the final challenge the moving platform will be provided with the marker in Fig.~\ref{fig:finalplatform}. In order to have a measurement update in the low-altitude EKF we have to implement a cross detector: instead of estimating the 4DoF pose of the platform from the AR-tag detection, we must be able to extract the same information from the cross mark.\\

To do so, standard line detection algorithm can be used. The symmetry of the cross can be an issues since it does not allow to detect a unique solution for the yaw orientation. On the other hand, once we are estimating the initial yaw angle we are able to detect correctly the changes in orientation, as far as between two consecutive measures the platform rotates a few degrees: we cannot distinguish between rotations of multiple of $90^o$, but we know that two measures close in time must provide close values, because the angular velocity of the platform is not really high.\\

From the point of view of our framework we can simply substitute the detection module. This new detector should provide the same data of the AR-tag detector used in this thesis, and so it can directly be used as update step on the EKF already implemented.\\
